\documentclass[uplatex, a4j]{jsarticle}

%option
\usepackage[dvipdfmx]{graphicx}
\usepackage[dvipdfmx]{color}
\usepackage{here}
\usepackage{url}
\usepackage{comment}

% 表紙
\title{BOLD fMRI解析の概要}
\author{βshort}
%\date{2019年 10月}

%\twocolumn[
%  \begin{@twocolumnfalse}
%    \maketitle
%    \begin{abstract}
%
%    \end{abstract}
%  \end{@twocolumnfalse}
%  ]

\begin{document}
\maketitle

\begin{abstract}
  BOLD fMRI解析の概要
\end{abstract}

% 本文
\section{はじめに}


\section{BOLD fMRIの原理}


\section{神経細胞と血流動態反応、そしてBOLD効果}
fMRI(functional Magnetic Resonance Imaging)の基礎は、BOLD(Blood Oxygenation Level Dependent) Responsesである。
BOLD Responsesは、Positive BOLD Responses(PBR)と、Negative BOLD Responses(NBR)に分けられる。
PBRは、実験課題で活性化(activated)する(fMRI値が増加する)反応である。
一方、NBRは、実験課題で不活性化(deactivated)する(fMRI値が減少する)反応である。

PBRは、神経活動の増加に伴う局所のdHbの減少を反映している。(血流量増大によるHbの過剰な流入)
NBRは、イニシャルディップはPBRの前に生じる短時間の信号低下で、血流量の増加に先立ち、dHbが一時的に蓄積するため生じると理解されている。
また、持続性のNBRも生じる。この答えは未だに完全には解明されていない。

\section{BOLD fMRI解析の基本}

\section{state-of-the-art}





\end{document}
