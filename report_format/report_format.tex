%欧文
\documentclass[uplatex, a4j]{jarticle}
\usepackage[dvipdfmx]{graphicx}
\usepackage{comment}
\usepackage{here}
\usepackage[top=20truemm,bottom=20truemm,left=20truemm,right=20truemm]{geometry}

\begin{document}
% 表紙
\title{レポートフォーマット}
\author{βshort}
\date{}

\maketitle

% 本文
\section{はじめに}

\section{ストーリー重視の「報告・相談型」レポート}
研究開発の過程をそのままストーリー化し、報告・相談をしやすくするフォーマット。

→ 「進捗報告」・「論文・技術報告」

1. 背景・前提

従来どうだったのか?
前回まで何が進んだのか?
何が前提となっているのか?

2. 課題

今、直面している課題は何か?
なぜそれを課題と捉えているのか?
課題に対する仮説は何か?

3. 手段・アプローチ

どう解決しようとしているか?
なぜその手段を採用するのか?
それはどんな意味を持つのか?

4. 効果・結論

結果から何が言えるのか?
なぜそれが言えるのか?
次はどうするつもりか?

\subsection{進捗報告:例}
前回は〜をした。
問題点は、〜であった。
そこで、今回は〜をした。
理由は〜だからである。
その結果〜ということがわかった。
理由は〜と考えられる
〜まとめ〜


\section{提案型}
最初に効果を提示し「いつまでに、こういうことが実現できる」と主張する。

→ 「企画書」・「プレゼン資料」

1. 効果

2. 結論

3. 背景・前提

4. 課題

5. 手段・アプローチ

\subsection{企画書:例}

1. 社会・ビジネス的な効果
2. 技術的な効果
3. 完了要件
4. 背景
5. 社会・ビジネス上での課題
6. 技術的な課題
7. 課題の解決方式
8. スケジュール概要
9. 必要なリソース

\section{論文}
1. 先行研究
関連する先行研究を紹介し、本研究のオリジナリティを説明する
2. 課題
本研究で解決したい課題
3. 手段
課題を解決するためのアプローチ
4. 結果
手法によって得られた結果を提示し、課題がどこまで解決できたかを説明する
5. 考察
そのような結果を得た理由を検討する
6. 結論
1~5のプロセスを要約し、今後の課題を示す。

\section{チェックポイント}

従来技術の把握は正しいか?

その課題が従来技術では解決できない原因に説得力はあるか?

基礎との差分「のみ」が抽出できているか?

課題と結論が裏返しの関係を満たしているか?

その結論が導ける理由に説得力はあるか?

目的に応じた構成を選択出来ているか?

\section{優れた課題の条件}
1. 解決の基礎が存在する(課題を解決可能とする条件が全て揃っている)

先行技術を正確に把握することが重要
「正しく理解された基礎」が「全て明らかになっている」ことが、最初に満たすべき条件


2. 解決の結果が次の展開の基礎となる

理想と現実とのギャップを直線で捉える。順に「問題を克服する」

3. 優れた仮説を生む



%参考文献
\begin{thebibliography}{3}
  \bibitem{文献1}藤田肇, (2019), "成果を生み出すテクニカルライティングトップエンジニア・研究者が実践する思考整理法", 株式会社技術評論社
\end{thebibliography}

\end{document}
