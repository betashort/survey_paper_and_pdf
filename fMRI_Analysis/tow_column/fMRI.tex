%欧文
\documentclass[uplatex, a4j, twocolumn]{article}

\begin{document}
% 表紙
\title{Review: fMRI Analysis}
\author{βshort}
% \date{}

\twocolumn[
  \begin{@twocolumnfalse}
    \maketitle
    \begin{abstract}
      functional MRI(fMRI) Analysisについてまとめる.
    \end{abstract}
  \end{@twocolumnfalse}
  ]
% 本文
%\chapter{序論}
\section{fMRIとは}
fMRI(functional Magnetic Resonance Imaging)

\section{fMRIの基礎}

\subsection{BOLD効果}
fMRIの基礎は、BOLD(Blood Oxygenation Level Dependent) Responsesである。

BOLD Responsesは、Positive BOLD Responses(PBR)と、Negative BOLD Responses(NBR)に分けられる。
PBRは、実験課題で活性化(activated)する(fMRI値が増加する)反応である。
一方、NBRは、実験課題で不活性化(deactivated)する(fMRI値が減少する)反応である。

PBRは、神経活動の増加に伴う局所のdHbの減少を反映している。(血流量増大によるHbの過剰な流入)
NBRは、イニシャルディップはPBRの前に生じる短時間の信号低下で、血流量の増加に先立ち、dHbが一時的に蓄積するため生じると理解されている。
また、持続性のNBRも生じる。この答えは未だに完全には解明されていない。
\subsubsection{正のBOLD効果}
\subsubsection{負のBOLD効果}
負のBOLD効果について
[2]"Activation and deactivation in blood oxygenation level dependent functional magnetic resonance imaging"


[3]"Interpreting Deactivations in Neuroimaging"

[4]Evidence that the negative BOLD response is neuronal in origin:A simultaneous EEG-BOLD-CBF studyin humans"

抑制性の介在ニューロン
[9]"Organizing principles for a diversity of GABAergic interneurons and synapses in the neocortex"

\subsection{fMRIデータの前処理}

\subsection{fMRIの実験デザイン}
ブロックデザイン/事象関連デザイン/
\subsection{fMRIの解析手法}
Statistics Parametric Mapping(SPM)
\subsection{fMRIの解析ソフト}
SPM/FSL/FreeSurfer
\section{fMRIの応用: 活用事例}
臨床など

\subsection{実験課題について}
[10]"What is that little voice inside my head?Inner speech phenomenology, its role in cognitive performance, and its relation to self-monitoring"


\subsection{側頭葉の介在ニューロンの比率}
[17]"Colocalization of calbindin D-28k, calretinin, and GABA immuroreactivities in neurons of the human temporal cortex"

\section{参考文献}
